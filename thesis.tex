% Vorlage für Masterabreit am Institut für Psychologie Graz
% Idee/Umsetzung: Mag. Dr. Karl Koschutnig / MRI LAb Graz
% Bei Fragen karl.koschutnig@uni-graz.at

\documentclass[man]{apa7}
% Load the settings


% === Hier alle Angaben zum Titelblatt ausfüllen ===
\newcommand{\thesisauthor}{Susi Sorglos}
\newcommand{\thesistitle}{Sitzt zu hause und föhnt ihr Haar}
\newcommand{\thesissubtitle}{Schabba du und schabba da}
\newcommand{\thesisadvisor}{Prof. Dr. Harry Hirsch}
\newcommand{\thesisinstitute}{Institut für Psychologie}
\newcommand{\thesisdate}{21.03.1917}


% apa_template_settings.tex
% Contains only preamble settings for the thesis

% Load packages and configure settings
\usepackage[utf8]{inputenc}
\usepackage{setspace}
\usepackage{graphicx}
\usepackage{csquotes}
\usepackage[american]{babel}
\usepackage[style=apa,sortcites=true,sorting=nyt,backend=biber]{biblatex}
\DeclareLanguageMapping{american}{american-apa}
\addbibresource{bibliography.bib}

% Header settings
\pagestyle{fancy}
\fancyhf{} % Clear header and footer
\fancyhead[L]{\shorttitle} % Running title on the left
\fancyhead[R]{\thepage} % Page number on the right
\renewcommand{\headrulewidth}{0pt} % Remove horizontal bar

% Document-wide formatting
\doublespacing
 

\begin{document}
\maketitlepage

% Abstract
\newpage
\begin{abstract}

This is the abstract section. Hallo and It provides a summary of the thesis, including the purpose, methods, results, and conclusions. Typically, this section should be no longer than 250 words.

\vspace{2mm}

\noindent\textbf{Keywords:} \textit{Keyword1, Keyword2}

\vspace{5mm}

\end{abstract}

% Kurzfassung
\newpage
\section*{Kurzfassung}
Dies ist die Kurzfassung der Arbeit. Sie fasst die Ziele, Methoden, Ergebnisse und Schlussfolgerungen zusammen.

% Danksagung
\newpage
\section*{Danksagung}
Ich möchte mich bei allen bedanken, die mich während der Erstellung dieser Arbeit unterstützt haben.

% Declaration
\newpage
\section*{Eidesstattliche Erklärung}
Hiermit erkläre ich, dass ich die vorliegende Arbeit selbstständig und nur mit den angegebenen Hilfsmitteln angefertigt habe.

% Table of Contents
\newpage
\tableofcontents
\newpage

% Chapters
\newpage
\section{Introduction}
The first sentence should reference a fink paper \cite{Fink2021}
\lipsum[1-4] % Replace with actual content

\section{Literature Review}
\lipsum[5-8]

\section{Methods}
\lipsum[9-12]

\section{Results}
\lipsum[13-16]

\section{Discussion}
\lipsum[17-20]

% References
\newpage
\section*{References}
\printbibliography

% Appendices (if any)
\appendix
\section{Appendix Title}
\lipsum[21-23]

\end{document}
